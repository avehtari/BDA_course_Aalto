\documentclass[a4paper,11pt]{article}

\usepackage[pdftex]{graphicx}
%\usepackage{babel}
\usepackage[utf8]{inputenc}
\usepackage[T1]{fontenc}
%\usepackage[T1,mtbold,lucidacal,mtplusscr,subscriptcorrection]{mathtime}
\usepackage{times}
\usepackage{amsmath}
\usepackage{url}
\urlstyle{same}
\usepackage{enumerate}
\usepackage{parskip}
\usepackage{hyperref}
\usepackage{microtype}


% if not draft, smaller printable area makes the paper more readable
\topmargin -4mm
\oddsidemargin 0mm
\textheight 225mm
\textwidth 150mm

%\parskip=\baselineskip

\DeclareMathOperator{\E}{E}
\DeclareMathOperator{\Var}{Var}
\DeclareMathOperator{\var}{var}
\DeclareMathOperator{\Sd}{Sd}
\DeclareMathOperator{\sd}{sd}
\DeclareMathOperator{\Bin}{Bin}
\DeclareMathOperator{\Beta}{Beta}
\DeclareMathOperator{\Poisson}{Poisson}
\DeclareMathOperator{\betacdf}{betacdf}
\DeclareMathOperator{\Invchi2}{Inv-\chi^2}
\DeclareMathOperator{\logit}{logit}
\DeclareMathOperator{\N}{N}
\DeclareMathOperator{\U}{U}
\DeclareMathOperator{\tr}{tr}
\DeclareMathOperator{\trace}{trace}

% Horizontal line
\newcommand{\HRule}{\rule{\linewidth}{0.5mm}}

\pagestyle{empty}

\begin{document}
\thispagestyle{empty}

\section*{Bayesian data analysis -- assignment 2}

This exercise is related to Chapters 1 and 2.

The maximum amount of points from this assignment is 3. In addition to the correctness of the answers, the overall quality and clearness of the report is evaluated.

Report all results to a single, {\bf anonymous} *.pdf -file and return it to \href{peergrade.io}{peergrade.io}. Include also source code to the report (either as an attachment or as a part of the answer). By anonymity it is meant that the report should not contain your name or student number.

You may find an additional discussion about choosing priors by Andrew Gelman useful
\url{http://andrewgelman.com/2017/10/04/worry-rigged-priors/}.

\HRule

\vspace{1cm}

\subsection*{Inference for binomial proportion (Computer)}

Algae status is monitored in 274 sites at Finnish lakes and rivers.
The observations for the 2008 algae status at each site are presented
in file {\tt algae.txt} ('0': no algae, '1': algae present).
% In year 2008 blue-green algae was observed at 44 sites.
Let $\pi$ be the probability of a monitoring site having detectable
blue-green algae levels.

Use a binomial model for observations and a $\Beta(2,10)$ prior
for $\pi$ in Bayesian inference. Formulate Bayesian model likelihood
$p(y|\pi)$, prior $p(\pi)$, and the resulting posterior $p(\pi|y)$.
Here it is not necessary to derive the posterior distribution as it has already been done in the book.
Also it is not necessary to write out the distributions; it is sufficient to use label-parameter format, e.g.\ $\Beta(\cdot,\cdot)$. Although recommended, plotting is not required in this exercise.
Use your model to answer the following questions:
\begin{enumerate}[a)]
\item What can you say about the value of the unknown $\pi$ according
  to the observations and your prior knowledge? Summarize your results
  with a point estimate and an interval estimate.
\item What is the probability that the proportion of monitoring sites with detectable algae levels $\pi$ is smaller than $\pi_0=0.2$ that is known from historical records?
\item What assumptions are required in order to use this kind of a
  model with this type of data?
\item Make prior sensitivity analysis by testing a couple of different reasonable priors. Summarize the results by one or two sentences.
\end{enumerate}
Hint: With a conjugate prior, a closed form posterior is Beta form (see
equations in the book). Useful functions: {\tt dbeta}, {\tt pbeta}, {\tt qbeta} in R and {\tt pdf}, {\tt cdf} and {\tt ppf} from {\tt scipy.stats.beta} in Python.



\end{document}

%%% Local Variables:
%%% mode: latex
%%% TeX-master: t
%%% End:
