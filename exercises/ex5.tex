\documentclass[a4paper,11pt]{article}

\usepackage[pdftex]{graphicx}
%\usepackage{babel}
\usepackage[utf8]{inputenc}
\usepackage[T1]{fontenc}
%\usepackage[T1,mtbold,lucidacal,mtplusscr,subscriptcorrection]{mathtime}
\usepackage{times}
\usepackage{amsmath}
\usepackage{url}
\usepackage{enumerate}
\usepackage{parskip}
\usepackage[colorlinks,urlcolor=black]{hyperref}
\usepackage{microtype}

% if not draft, smaller printable area makes the paper more readable
\topmargin -4mm
\oddsidemargin 0mm
\textheight 225mm
\textwidth 150mm

%\parskip=\baselineskip

\DeclareMathOperator{\E}{E}
\DeclareMathOperator{\Var}{Var}
\DeclareMathOperator{\var}{var}
\DeclareMathOperator{\Sd}{Sd}
\DeclareMathOperator{\sd}{sd}
\DeclareMathOperator{\Bin}{Bin}
\DeclareMathOperator{\Beta}{Beta}
\DeclareMathOperator{\Poisson}{Poisson}
\DeclareMathOperator{\betacdf}{betacdf}
\DeclareMathOperator{\Invchi2}{Inv-\chi^2}
\DeclareMathOperator{\logit}{logit}
\DeclareMathOperator{\N}{N}
\DeclareMathOperator{\U}{U}
\DeclareMathOperator{\tr}{tr}
\DeclareMathOperator{\trace}{trace}

\pagestyle{empty}

\begin{document}
\thispagestyle{empty}

\section*{Bayesian data analysis -- exercise 5}

This assignment is related to Chapters 10 and 11.

The maximum amount of points from this assignment is 6. In addition to the correctness of the answers, the overall quality and clearness of the report is evaluated.

Report all results to a single, {\bf anonymous} *.pdf -file and return it to \href{peergrade.io}{peergrade.io}. Include also source code to the report (either as an attachment or as a part of the answer). By anonymity it is meant that the report should not contain your name or student number.

\vspace{1cm}





\subsection*{1. Generalized linear model: Bioassay with Metropolis (6 points)}

Metropolis algorithm: Replicate the computations for the bioassay
example of section 3.7 (BDA3) using the Metropolis algorithm. Be sure
to define your starting points and your jumping rule (proposal distribution). Run the
simulations long enough for approximate convergence.


\subsubsection*{Information and hints}

\begin{itemize}
\item Use uniform prior as in the book $p(\alpha,\beta)\propto 1$.
\item Use a simple proposal distribution, there is no need to try to find optimal proposal (but remember to report the one you used). Efficient proposals are dicussed in BDA3 p. 295--297 (not part of the course). In real-life a pre-run could be made with an automatic adaptive control to adapt the proposal distribution.
\item Metropolis is a simple algorithm, you do not need many lines of
  code for it (less than ten lines). %Insert your code into Matlab template ex11\_2.m.
\item Compute with log-densities. Reasons are explained on page 261 (BDA3). Remember that $\log(a/b)=\log(a)-\log(b)$. For your convenience we have provided functions for each language that will evaluate the log-posterior for given $\alpha$ and $\beta$ (see {\tt bioassaylp.R} and {\tt bioassaylp.py}).
\item Use $\hat{R}$ (see {\tt psrf.m}) for convergence analysis. Remember to remove the warm-up samples. Include in the report the number of chains used, the starting points, the number of samples generated from each chain, and the warm-up length. Report also the $\hat R$ values for $\alpha$ and $\beta$ and draw conclusions about the convergence of the chains.
\textbf{This means that you should briefly explain how to interpret the obtained} $\hat R$ \textbf{values}.
\item Plot the samples for $\alpha$ and $\beta$ (scatter plot) and compare to the Figure~3.3b in BDA3 to verify that your code works. You can also compute and visualize the density in a grid (as in Fig.~3.3a) but this is not mandatory.
\end{itemize}






\end{document}

%%% Local Variables:
%%% mode: latex
%%% TeX-master: t
%%% End:
