\documentclass[a4paper,11pt]{article}

\usepackage[pdftex]{graphicx}
%\usepackage{babel}
\usepackage[utf8]{inputenc}
\usepackage[T1]{fontenc}
%\usepackage[T1,mtbold,lucidacal,mtplusscr,subscriptcorrection]{mathtime}
\usepackage{times}
\usepackage{amsmath}
\usepackage{url}
\usepackage{enumerate}
\usepackage{parskip}
\usepackage{hyperref}
\usepackage{microtype}


%\usepackage[dvips,bookmarks=false]{hyperref}
% \hypersetup{%
%   bookmarksopen=true,
%   bookmarksnumbered=true,
%   pdftitle={S-114.2601 Bayesilaisen mallintamisen perusteet},
%   pdfsubject={Kommentteja},
%   pdfauthor={Aki Vehtari},
%   pdfkeywords={bayesilainen todennäköisyysteoria, bayesilainen
%     päättely, bayesilaiset mallit, mallien analysointi,
%     laskennalliset menetelmät, Markov-ketju Monte Carlo},
%   pdfstartview={FitH -32768}
% }


% if not draft, smaller printable area makes the paper more readable
\topmargin -4mm
\oddsidemargin 0mm
\textheight 225mm
\textwidth 150mm

%\parskip=\baselineskip

\DeclareMathOperator{\E}{E}
\DeclareMathOperator{\Var}{Var}
\DeclareMathOperator{\var}{var}
\DeclareMathOperator{\Sd}{Sd}
\DeclareMathOperator{\sd}{sd}
\DeclareMathOperator{\Bin}{Bin}
\DeclareMathOperator{\Beta}{Beta}
\DeclareMathOperator{\Poisson}{Poisson}
\DeclareMathOperator{\betacdf}{betacdf}
\DeclareMathOperator{\Invchi2}{Inv-\chi^2}
\DeclareMathOperator{\logit}{logit}
\DeclareMathOperator{\N}{N}
\DeclareMathOperator{\U}{U}
\DeclareMathOperator{\tr}{tr}
\DeclareMathOperator{\trace}{trace}

\pagestyle{empty}

\begin{document}
\thispagestyle{empty}

\section*{Bayesian data analysis -- exercise 4}

This exercise is related to Chapters 3 and 10.

The maximum amount of points from this assignment is 6. In addition to the correctness of the answers, the overall quality and clearness of the report is evaluated.

Report all results to a single, {\bf anonymous} *.pdf -file and return it to \href{peergrade.io}{peergrade.io}. Include also source code to the report (either as an attachment or as a part of the answer). By anonymity it is meant that the report should not contain your name or student number.

\vspace{1cm}


\subsection*{Generalized linear model: Bioassay with importance sampling}


In the bioassay example (Chapter 3 in the book), replace the uniform prior density by a joint normal prior distribution on $(\alpha, \beta)$, with $\alpha \sim N(0,2^2), \beta \sim N(10,10^2)$, and $\mathrm{corr}(\alpha, \beta)=0.5$.

Repeat parts of the computations and plots of Section 3.7 with this new prior distribution and using importance sampling and resampling.

  \begin{itemize}
  \item Calculate the posterior density in a grid of points around the prior ($\alpha$: $0 \pm 4$, $\beta$: $10 \pm 20$) and plot a heatmap of the density.
\item Sample draws of $\alpha$ and $\beta$ from the prior distribution.
\item Compute the importance ratios for each draw (target distribution is the posterior).
\item Compute and report the effective sample size $S_{\text{eff}}$. If $S_{\text{eff}}$ is less than 1000, repeat the importance sampling with more draws.
\item Compute the posterior mean using importance sampling and report it.
\item Use importance resampling to obtain a posterior sample of size 1000 of $\alpha$ and $\beta$.
\item Plot a scatterplot of the obtained posterior sample and compare that to the heatmap plotted earlier.
\item Report an estimate for $p(\beta > 0|x, n, y)$, that is, the probability that the drug is harmful.
\item Draw a histogram of the draws from the posterior distribution of the LD50 conditional on $\beta$ > 0 (see Figure 3.4).
  \end{itemize}
Hints
\begin{itemize}
\item See {\tt demo3\_6.R} / {\tt demo3\_6.ipynb} and {\tt demo10\_3.R} / {\tt demo10\_3.ipynb}
\item Compute the posterior density and importance ratios as logarithms
\item Scale the log-posterior by subtracting its maximum value before
  exponentiating
  \item Normalize the posterior and importance ratios to sum to 1
  \item When computing the importance ratios, you will have to take logarithms of very small numbers, which may round to zero. If this happens, replace the small numbers with {\tt 1e-12} to prevent {\tt log(0) = NaN} (not a number).
\end{itemize}






\end{document}

%%% Local Variables:
%%% mode: latex
%%% TeX-master: t
%%% End:
