\documentclass[t]{beamer}
%\documentclass[finnish,english,handout]{beamer}

% Uncomment if want to show notes
% \setbeameroption{show notes}

\mode<presentation>
{
  \usetheme{Copenhagen}
  % oder ...

  %\setbeamercovered{transparent}
  % oder auch nicht
}


\usepackage[latin1]{inputenc}
\usepackage{times}
\usepackage{url}
\urlstyle{same}
\usepackage{hyperref}
% \usepackage{enumerate}
% \usepackage{parskip}
% \usepackage{enumitem}% http://ctan.org/pkg/enumitem

\hypersetup{%
  bookmarksopen=true,
  bookmarksnumbered=true,
  pdftitle={Stan},
  pdfsubject={Bayesian data analysis},
  pdfauthor={Aki Vehtari},
  pdfkeywords={},
  pdfstartview={FitH -32768},
  colorlinks=true,
  linkcolor=navyblue,
  citecolor=navyblue,
  filecolor=navyblue,
  urlcolor=navyblue
}


% \definecolor{hutblue}{rgb}{0,0.2549,0.6784}
% \definecolor{midnightblue}{rgb}{0.0977,0.0977,0.4375}
% \definecolor{hutsilver}{rgb}{0.4863,0.4784,0.4784}
% \definecolor{lightgray}{rgb}{0.95,0.95,0.95}
% \definecolor{section}{rgb}{0,0.2549,0.6784}
% \definecolor{list1}{rgb}{0,0.2549,0.6784}
\definecolor{forestgreen}{rgb}{0.1333,0.5451,0.1333}
\definecolor{navyblue}{rgb}{0,0,0.5}
\renewcommand{\emph}[1]{\textcolor{navyblue}{#1}}

%\graphicspath{{./figs/}}

\pdfinfo{
  /Title      (Bayesian data analysis)
  /Author     (Aki Vehtari) %
  /Keywords   (Bayesian probability theory, Bayesian inference, Bayesian data analysis)
}


\parindent=0pt
\parskip=8pt
\tolerance=9000
\abovedisplayshortskip=0pt

\setbeamertemplate{navigation symbols}{}
\setbeamertemplate{headline}[default]{}
\setbeamertemplate{headline}[text line]{\insertsection}
\setbeamertemplate{footline}[frame number]


\title[]{Bayesian data analysis}
\subtitle{}

\author{Aki Vehtari}

\institute[Aalto]{}

\begin{document}

\begin{frame}

  {\Large\color{navyblue} Project work}
  
  \begin{itemize}
  \item Choose a data set and make all the steps of Bayesian data
    analysis workflow listed below
  \item Project outcome is a R or Python notebook similar to notebooks
    in (many of these notebooks don't have all the required parts)
    \begin{itemize}
    \item BDA R demos \url{https://github.com/avehtari/BDA_R_demos/tree/master/demos_rstan}
    \item BDA Python demos \url{https://github.com/avehtari/BDA_py_demos/tree/master/demos_pystan}
    \item Stan case studies \url{http://mc-stan.org/users/documentation/case-studies.html}
    \item StanCon case studies \url{http://mc-stan.org/users/documentation/case-studies.html}
      (some of these notebooks are for a bigger projects, but reflect still the basic idea of a notebook presentation)
    \end{itemize}
  \item The submitted notebooks need to illustrate the knowledge of the
    Bayesian workflow.
  \end{itemize}
\end{frame}

\begin{frame}
  
  {\Large\color{navyblue} Project work}
  
  \begin{itemize}
  \item The notebooks have to include
    \begin{itemize}
    \item Description of the data, and the analysis problem
    \item Description of at least two models, for example:
      \begin{itemize}
      \item non-hierarchical and hierarchical
      \item linear and non-linear
      \item variable selection with many models
    \end{itemize}
    \item Informative or weakly informative priors, and description of the prior choices
    \item Stan code
    \item How Stan model is run
    \item Convergence diagnostics (Rhat, divergences, neff)
    \item Posterior predictive checking
    \item Model comparison (e.g. with loo)
    \item Predictive performance assessment if applicable (e.g. classification
      accuracy)
    \item Sensitivity analysis with respect to prior choices
    \item Discussion of problems, and potential improvements 
      \begin{itemize}
      \item It is possible that your model or inference is not perfect, but a better model would require substantial work. Then it's ok that you report the problems found (using the various diagnostics discussed in the course) and describe possible improvements.
      \end{itemize}
    \end{itemize}
  \end{itemize}
\end{frame}

\begin{frame}
  
  {\Large\color{navyblue} Project work}
  
  \begin{itemize}
  \item You can re-use of code and text from existing case studies
    \begin{itemize}
    \item Just report what did you re-use
    \item Acknowledge the original authors
    \item Include the original copyright licence
      \begin{itemize}
      \item CC-BY or CC-BY-NC is common for text
        \url{https://creativecommons.org/licenses/}
      \item BSD-3 is common for code
        \url{https://opensource.org/licenses/BSD-3-Clause}
      \end{itemize}
    \item Don't use improper priors even if some case study has improper priors
  \end{itemize}
  \item You can use BRMS to create Stan code, but do not limit yourself
    to BRMS models if changes would make a better model
  \end{itemize}
\end{frame}

\begin{frame}
  
  {\Large\color{navyblue} Oral presentation}
  
  \begin{itemize}
  \item During evaluation week 50
  \item Each project needs to be presented in addition to submitting the notebook
  \item The presentation should be high level but sufficiently detailed information should be readily available to facilitate answering questions from the audience
  \item Within each session, about four groups will be presenting
  \item For 1-2 person groups, the presentation should be 10 minutes
  \item For 3 person groups, the presentation should be 15 minutes 
  \item Afterwards, questions will be asked first by other students and then by two attending TAs for about 5 to 10 minutes
  \item Grading of the presentation will be done by the two TAs using standardized grading instructions
  \end{itemize}
\end{frame}

\begin{frame}
  
  {\Large\color{navyblue} Some special topics}

  \begin{itemize}
  \item Improve R demos
  \item Dynamic HMC demo in R or Python
  \end{itemize}
  
\end{frame}

\begin{frame}
  
  {\Large\color{navyblue} Some ideas for data sets}
  
  \begin{itemize}
  \item How do People Type on Mobile Devices? Observations from a
    Study with 37,000 Volunteers \url{https://userinterfaces.aalto.fi/typing37k/}
  \item Laptop multitasking hinders classroom learning for both users
    and nearby peers
    \url{http://www.sciencedirect.com/science/article/pii/S0360131512002254}
  \item Arctic sea ice shrinking \url{https://www.nytimes.com/interactive/2017/09/22/climate/arctic-sea-ice-shrinking-trend-watch.html}
  \item Finnish weather statistics \url{https://en.ilmatieteenlaitos.fi/statistics-from-1961-onwards}
  \item R datasets \url{https://vincentarelbundock.github.io/Rdatasets/datasets.html}
  \item Vanderbilt Biostatistics \url{http://biostat.mc.vanderbilt.edu/wiki/Main/DataSets}
  \item Probaly better to *not* have a data set
    \begin{itemize}
    \item with number of observation in millions
    \item machine vision task
    \end{itemize}
  \end{itemize}
\end{frame}

\begin{frame}
  
  {\Large\color{navyblue} Schedule}


  \begin{itemize}
  \item Register project group and topic by 4th November
    \begin{itemize}
    \item 2 person group preferred, 3 and 1 person groups allowed
    \item 3 person groups are expected to choose more difficult projects
	\item 2-3 person groups are highly recommended over 1 person groups. 2-3 person groups have priority when reserving presentation slots
    \end{itemize}
  \item During the week starting 4th November, start working on the
    project and if necessary talk with TAs (no new assignment on that
    week)
  \item Deadline end of week 49, 8 December
  \item Oral presentations during the evaluation week (week 50)
  \end{itemize}
  
\end{frame}

\end{document}

%%% Local Variables: 
%%% TeX-PDF-mode: t
%%% TeX-master: t
%%% End: 
