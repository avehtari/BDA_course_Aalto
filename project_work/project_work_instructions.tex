\documentclass[a4paper,11pt]{article}

\usepackage[pdftex]{graphicx}
\usepackage[utf8]{inputenc}
\usepackage[T1]{fontenc}
\usepackage{times}
\usepackage{amsmath}
\usepackage[hyphens]{url}
\urlstyle{same}
\usepackage{enumerate}
\usepackage{parskip}
\usepackage[colorlinks,urlcolor=navyblue]{hyperref}
\usepackage{microtype}
\usepackage{enumitem}% http://ctan.org/pkg/enumitem

\usepackage{color}
\definecolor{navyblue}{rgb}{0,0,0.5}

% if not draft, smaller printable area makes the paper more readable
\topmargin -4mm
\oddsidemargin 0mm
\textheight 225mm
\textwidth 150mm

%\parskip=\baselineskip

\pagestyle{empty}

\begin{document}
\thispagestyle{empty}

\section*{Instructions -- project work}
\begin{itemize}[noitemsep,topsep=0pt]
\item Choose a data set and make all the steps of Bayesian data
  analysis workflow listed below
\item Project outcome is a Python or R notebook similar to notebooks in
\begin{itemize}[noitemsep,topsep=0pt]
  \item BDA R demos \url{https://github.com/avehtari/BDA_R_demos/tree/master/demos_rstan}
  \item BDA Python demos \url{https://github.com/avehtari/BDA_py_demos/tree/master/demos_pystan}
  \item Stan case studies \url{http://mc-stan.org/users/documentation/case-studies.html}
  \item StanCon case studies \url{http://mc-stan.org/users/documentation/case-studies.html}
  (some of these notebooks are for a bigger projects, but reflect still the basic idea of a notebook presentation)
\end{itemize}
\item The submitted notebooks need to illustrate the knowledge of the
  Bayesian workflow.
\item The notebooks have to include
\begin{itemize}
  \item Description of the data, and the analysis problem
  \item Description of the model
  \item Description of the prior choices
  \item Stan code
  \item How Stan model is run
  \item Convergence diagnostics (Rhat, divergences, neff)
  \item Posterior predictive checking
  \item Model comparison (e.g. with loo)
  \item Predictive performance assessment if applicable (e.g. classification
    accuracy)
  \item Potentially sensitivity analysis
  \item Discussion of problems, and potential improvements 
\end{itemize}
\end{itemize}

\section*{Peergrade rubric}

Part of the questions are used to check that the minimal requirements
of the project work are included. Most of the questions are for giving
feedback to other students. The received feedback and your response to
that will be discussed in the evaluation meeting.
%Peergrade score you receive is not you final grade for the project work.

\begin{itemize}
\item Can you open the notebook?
\begin{itemize}
\item yes
\item no
\end{itemize}

\item Is there an introduction? 
  \begin{itemize}
  \item There is no clear introduction
  \item The introduction touches on the main topic
  \item The introduction states the main topic and provides an overview of the notebook
  \item The introduction is inviting, presents an overview of the
    notebook. Information is relevant and presented in a logical
    order.
\end{itemize}

\item Do you have any suggestions on how to improve the introduction?

\item Is there a conclusion? 
  \begin{itemize}
  \item There is no clear conclusion
  \item A conclusion is included
  \item The conclusion is clear
  \end{itemize}
  
Describe in your own words what is the main conclusion of the data analysis in this notebook?

\item The structure and organization of the notebook
  \begin{itemize}
  \item The notebook lacks a clear data analysis story
  \item The notebook attempts to tell a coherent data analysis story but lacks some focus and clarity.
  \item The notebook presents a clear cohesive data analysis story
  \item The notebook presents a clear cohesive data analysis
    story, which is enjoyable to read
  \end{itemize}
  
\item Overall, what did you think of the structure and organization of the
notebook? Name at least one way your peer could improve structure and
organization.

\item Accuracy of use of statistical terms
  \begin{itemize}
  \item There are numerous errors in use statistical terms
  \item There are some errors in use of statistical terms
  \item Statistical terms are used accurately but sometimes lack clarity
  \item Statistical terms are used accurately and with clarity
  \end{itemize}
  
\item  Description of the data, and the analysis problem
  \begin{itemize}
  \item yes
  \item no
  \item Did you get a sense of what is the data and the analysis problem when they were first introduced? Where and how might the author make the model description more clear?
\end{itemize}

\item Description of the model
  \begin{itemize}
  \item yes
  \item no
  \item Did you get a sense of what is the model? Where and 
  how might the author make the model description more clear?
\end{itemize}

\item Description of the prior choices
  \begin{itemize}
  \item No priors
  \item Priors listed but not justified
  \item Priors are listed and justified
  \end{itemize}
  
\item Is Stan code included?
  \begin{itemize}
  \item yes
  \item no
  \end{itemize}

  
\item Is code for how Stan model is run included?
  \begin{itemize}
  \item yes
  \item no
  \end{itemize}

  \item Is required convergence diagnostics (Rhat, divergences, neff) included?
    \begin{itemize}
    \item No convergence diagnostics at all
  \item Not all required diagnostics are included
  \item Required convergence diagnostic results shown but not discussed
  \item Required onvergence diagnostic results shown and maning of the results is discussed
  \end{itemize}

\item Is there posterior predictive checking?
  \begin{itemize}
  \item yes
  \item no
  \end{itemize}

\item Is there a discussion of problems and potential improvements ?
  \begin{itemize}
  \item yes
  \item no
  \end{itemize}

\item Choose something you like about the notebook and explain why you like it. 

\item If you were to go back and redo your own notebook after reading this submission, what would you change?

\item If the student were to complete this project work again, what could they change, to make it overall better?
\end{itemize}





\end{document}

%%% Local Variables:
%%% mode: latex
%%% TeX-master: t
%%% End:
