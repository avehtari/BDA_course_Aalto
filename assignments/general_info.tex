
\subsubsection*{General information}


\begin{itemize}
\itemsep0em
\item The recommended tool in this course is R (with the IDE R-Studio). You can download R \href{https://cran.r-project.org/}{\textbf{here}} and R-Studio \href{https://www.rstudio.com/products/rstudio/download/}{\textbf{here}}. There are tons of tutorials, videos and introductions to R and R-Studio online. You can find some initial hints from \href{https://education.rstudio.com/}{\textbf{RStudio Education pages}}.
\item Instead of installing R and RStudio on you own computer, see \href{https://avehtari.github.io/BDA_course_Aalto/FAQ.html#How_to_use_R_and_RStudio_remotely}{\textbf{how to use R and RStudio remotely}}.
\item When working with R, we recommend writing the report using R markdown and the provided \href{https://raw.githubusercontent.com/avehtari/BDA_course_Aalto/master/templates/assignment_template.rmd}{\textbf{R markdown template}}. The remplate includes the formatting instructions and how to include code and figures.
\item Instead of R markdown, you can use other software to make the PDF report, but the the same instructions for formatting should be used. These instructions are available also in \href{https://raw.githubusercontent.com/avehtari/BDA_course_Aalto/master/templates/assignment_template.pdf}{\textbf{the PDF produced from the R markdown template}}.
\item  Report all results in a single, {\bf anonymous} *.pdf -file and return it to \href{peergrade.io}{\textbf{peergrade.io}}.
\item The course has its own R package \texttt{aaltobda} with data and functionality to simplify coding. To install the package just run the following (upgrade="never" skips question about updating other packages):
\begin{enumerate}
\item \texttt{install.packages("remotes")}
\item \texttt{remotes::install\_github("avehtari/BDA\_course\_Aalto", \\ subdir = "rpackage", upgrade="never")}
\end{enumerate}
\item Many of the exercises can be checked automatically using the R package \\ \texttt{markmyassignment}. Information on how to install and use the package can be found \href{https://cran.r-project.org/web/packages/markmyassignment/vignettes/markmyassignment.html}{\textbf{here}}. There is no need to include \texttt{markmyassignment} results in the report.
\item Recommended additional self study exercises for each chapter in BDA3 are listed in the course web page.
\item Common questions and answers regarding installation and technical problems can be found in \href{https://avehtari.github.io/BDA_course_Aalto/FAQ.html}{Frequently Asked Questions (FAQ)}.
\item Deadlines for all assignments can be found on the course web page and in peergrade. You can set email alerts for trhe deadlines in peergrade settings.
\item You are allowed to discuss assignments with your friends, but it is not allowed to copy solutions directly from other students or from internet. You can copy, e.g., plotting code from the course demos, but really try to solve the actual assignment problems with your own code and explanations. Do not share your answers publicly. Do not copy answers from the internet or from previous years. We compare the answers to the answers from previous years and to the answers from other students this year. All suspected plagiarism will be reported and investigated. See more about the \href{https://into.aalto.fi/display/ensaannot/Aalto+University+Code+of+Academic+Integrity+and+Handling+Violations+Thereof}{\textbf{Aalto University Code of Academic Integrity and Handling Violations Thereof}}.
\item Do not submit empty PDFs or almost empty PDFs as these are just harming the other students as they can't do peergrading for the empty or almost empty submissions. Violations of this rule will be reported and investigated in the same way was plagiarism.
\item If you have any suggestions or improvements to the course material, please post in the course chat feedback channel, create an issue, or submit a pull request to the public repository!
\end{itemize}
