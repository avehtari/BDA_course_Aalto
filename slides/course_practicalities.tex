\documentclass[finnish,english]{beamer}
%\documentclass[finnish,english,handout]{beamer}

% Uncomment if want to show notes
%\setbeameroption{show notes}

\mode<presentation>
{
  \usetheme{Warsaw}
  % oder ...

  %\setbeamercovered{transparent}
  % oder auch nicht
}


%\usepackage[pdftex]{graphicx}
\usepackage[T1]{fontenc}
\usepackage[latin1]{inputenc}
%\usepackage[T1,mtbold,lucidacal,mtplusscr,subscriptcorrection]{mathtime}
\usepackage{times}
\usepackage{epic,epsfig}
\usepackage{subfigure,float}
\usepackage{amsmath,amsfonts,amssymb}
\usepackage{inputenc}
\usepackage{babel}
%\usepackage{euscript}
\usepackage{afterpage}
%\usepackage{picinpar}
%\usepackage{array,longtable}
\usepackage{url}
\urlstyle{same}
\usepackage{eufrak}
\usepackage{amsbsy}
\usepackage{eucal}
\usepackage{rotating}

\usepackage{natbib}
\bibliographystyle{apalike}

% \definecolor{hutblue}{rgb}{0,0.2549,0.6784}
% \definecolor{midnightblue}{rgb}{0.0977,0.0977,0.4375}
% \definecolor{navyblue}{rgb}{0,0,0.5}
% \definecolor{hutsilver}{rgb}{0.4863,0.4784,0.4784}
% \definecolor{lightgray}{rgb}{0.95,0.95,0.95}
% \definecolor{section}{rgb}{0,0.2549,0.6784}
% \definecolor{list1}{rgb}{0,0.2549,0.6784}
% \renewcommand{\emph}[1]{\textcolor{navyblue}{#1}}

\graphicspath{../luku1}

\pdfinfo{
          /Title      (Bayesian data analysis)
          /Author     (Aki Vehtari) %
          /Keywords   (Bayesian probability theory, Bayesian inference, Bayesian data analysis)
}


\parindent=0pt
\parskip=8pt
\tolerance=9000
\abovedisplayshortskip=0pt

\setbeamertemplate{navigation symbols}{}
\setbeamertemplate{headline}[default]{}
\setbeamertemplate{headline}[text line]{\insertsection}
\setbeamertemplate{footline}[frame number]


\def\o{{\mathbf o}}
\def\t{{\mathbf \theta}}
\def\w{{\mathbf w}}
\def\x{{\mathbf x}}
\def\y{{\mathbf y}}
\def\z{{\mathbf z}}

\DeclareMathOperator{\E}{E}
\DeclareMathOperator{\Var}{Var}
\DeclareMathOperator{\var}{var}
\DeclareMathOperator{\Sd}{Sd}
\DeclareMathOperator{\sd}{sd}
\DeclareMathOperator{\Bin}{Bin}
\DeclareMathOperator{\Beta}{Beta}
\DeclareMathOperator{\logit}{logit}
\DeclareMathOperator{\N}{N}
\DeclareMathOperator{\U}{U}
\DeclareMathOperator{\BF}{BF}
%\DeclareMathOperator{\Pr}{Pr}
\def\euro{{\footnotesize \EUR\, }}
\DeclareMathOperator{\rep}{\mathrm{rep}}


% ============
% Otsikko sivu
% ============

\title[]{Bayesian data analysis}
\subtitle{Practical matters}

\author{Aki Vehtari}

\institute[Aalto University]{}

\begin{document}

\section{Course contents}


\begin{frame}
  \frametitle{Bayesian data analysis (Aalto fall 2018)}  %
  \framesubtitle{}
  \begin{itemize}
  \item Book: Gelman, Carlin, Stern, Dunson, Vehtari \& Rubin: Bayesian Data
    Analysis, Third Edition.
  \item Timetable: Lectures on Mondays at 14-16, TAs available Thursdays 12-16, Fridays 10-12
  \end{itemize}
 \begin{center}
   \includegraphics[width=3cm]{figs/BDA3.jpg}
 \end{center}

\end{frame}

\begin{frame}
  \frametitle{Bayesian data analysis}  %
  \framesubtitle{Pre-requisites}
  \begin{itemize}
  \item Basic terms of probability theory
    \begin{itemize}
    \item probability, probability density, distribution
    \item sum, product rule, and Bayes' rule
    \item expectation, mean, variance, median
    \end{itemize}
  \item Some algebra and calculus
  \item Basic visualisation techniques (R or Python)
    \begin{itemize}
    \item histogram, density plot, scatter plot
    \end{itemize}
  \end{itemize}

  These will be tested with the first assignment round

\end{frame}

\begin{frame}
  \frametitle{Bayesian data analysis}  %
  \framesubtitle{Course contents}
  \begin{itemize}
  \item Background (Ch 1)
  \item Single-parameter models (Ch 2)
  \item Multiparameter models (Ch 3)
  \item Computational methods (Ch 10)
  \item Markov chain Monte Carlo (Ch 11--12)
  \item Stan and probabilistic programming
  \item Hierarchical models (Ch 5)
  \item Model checking (Ch 6)
  \item Evaluating and comparing models (Ch 7)
  \item Decision analysis (Ch 9)
  \item Large sample properties and Laplace approximation (Ch 4)
  \item In addition you learn workflow for Bayesian data analysis
  \end{itemize}
  
\end{frame}

\begin{frame}
  \frametitle{Bayesian data analysis}  %

  \begin{itemize}
  \item Lectures describe basics and give broader overview
    \begin{itemize}
    \item part of lecture time for questions
     \item written material has all the details and self-study
       is possible
    \end{itemize}
  \item Supporting material, assignments and news in MyCourses
  \item Supporting material and assignments in \url{https://github.com/avehtari/BDA_course_Aalto}
    \begin{itemize}
    \item reading instructions and chapter notes
    \item demos
    \item slides
    \item links to additional material
    \end{itemize}
  \item R demos \url{https://github.com/avehtari/BDA_R_demos/}
  \item Python demos \url{https://github.com/avehtari/BDA_py_demos/}
  \end{itemize}

\end{frame}

% \begin{frame}
%   \frametitle{Bayesian data analysis (Aalto fall 2015)}  %
%   \framesubtitle{}
%   \begin{itemize}
%   \item This spring the course has been designed so that it's strongly
%     recommended to take both
%     \begin{itemize}
%     \item Becs-114.1311 Introduction to Bayesian Statistics, 3cr
%     \item Becs-114.5312 Work Course on Bayesian Analysis, 2cr
%     \end{itemize}
%     \pause
%   \item In the future these are combined to Bayesian data analysis, 5
%     cr
%   \end{itemize}

% \end{frame}

% \begin{frame}
%   \frametitle{Bayesian data analysis}  %
%   \framesubtitle{Pre-exam}
%   \begin{itemize}
%   \item 10th September
%     \begin{itemize}
%     \item A-K 8:15-9:30 T7/CS-building
%     \item L-R 10:15-11:30 T7/CS-building
%     \item S-� 12:15-13:30 Maari-C
%     \end{itemize}
%   \item Obligatory
%   \item The correct solutions will be explained in detail 14th
%     September 14-16 (pen and paper solutions explained instead of the
%     lecture) and 17th September 12-14 (computer solutions explained in
%     exercise session)
%   \item If you find both the pre-exam and the explanations of the
%     solutions difficult, you should learn the pre-requisites before
%     continuing on the course
%   \item Pre-exam helps us also to learn about the level of students
%   \end{itemize}


% \end{frame}

\begin{frame}
  \frametitle{Bayesian data analysis}  %
  \framesubtitle{Exercises}
  \begin{itemize}
  \item Exercises are given on PeerGrade (also available in git repo)
  \item Exercises are returend and graded on Peergrade
  \item R/Python simulation exercises
  \item Stan exercises (via R/Python)
    \begin{itemize}
    \item Stan is a probabilistic programming language implementing
      full Bayesian statistical inference
    \end{itemize}
  \end{itemize}
\end{frame}

\begin{frame}
  \frametitle{Bayesian data analysis}  %
  \framesubtitle{Computer exercises}%, total 60p, min 30p for passing}
  \begin{itemize}
  \item Basic visualisation techniques
  \item Binomial distribution -- Algae
  \item Normal distribution -- Windshield% (3p)
  \item Difference between binomials -- Treatment/control% (3p)
  \item Difference between normals -- Windshield% (3p)
  \item Generalized linear model (GLM) + grid sampling -- Bioassay % (6p)
  \item GLM + Metropolis + convergence diagnostics -- Bioassay% (6p)
  \item GLM + Bioassay + Stan% (6p)
  \item Linear model + Stan % (3p)
  \item Hierarchical model + Stan% (3p)
  \item Model seletion + Stan% (6p)
  %\item GLM + Bioassay + Laplace approximation (6p)
  %\item Decision making (6p)
  \end{itemize}

\end{frame}

\begin{frame}
  \frametitle{Bayesian data analysis}  %
  \framesubtitle{Example analyses}%, total 60p, min 30p for passing}
  \begin{itemize}
  \item Treatment/control
    \begin{itemize}
    \item randomize patients to treatment or control
    \item is the treatment effective?
    \end{itemize}
    \pause
  \item Continuous valued treatment
    \begin{itemize}
    \item randomize patients with different dosages
    \item which dosage is sufficient without too many side effects?
    \end{itemize}
    \pause
  \item Different effects for different patients?
    \begin{itemize}
    \item Is the treatment effect different for male/female, child/adult, light/heavy, ...
    \end{itemize}
  \end{itemize}

\end{frame}


% \begin{frame}
%   \frametitle{Bayesian data analysis}  %
%   \framesubtitle{Advanced exercise}
%   \begin{itemize}
%   \item Own data and model
%   \end{itemize}

% \end{frame}

\begin{frame}
  \frametitle{Bayesian data analysis}  %
  \framesubtitle{Assessment}
  \begin{itemize}
  \item Exercises (48p) and project work (24p\%)
  % \item Grades will be calculated as follows
     \begin{itemize}
  %   \item Exam points are scaled and the final grade depends 51\% on exam and 49\% on exercises
     \item Minimum of 50\% of points must be obtained from both the exam and the exercises.
  %   \item One extra point for the exercises if you report the time you spent on solving each set of exercises (these are collected to monitor the workload of the course and will not affect the final grade)
     \item Preliminary grade boundaries\\
       <50\%=0, 50\%-60\%=1, 60\%-70\%=2, 70\%-80\%=3, 80\%-90\%=4, >90\%=5
     \end{itemize}
  \end{itemize}

\end{frame}

% \begin{frame}
%   \frametitle{Bayesian data analysis}  %
%   \framesubtitle{Course material}
%   \begin{itemize}
%   \item The course book is the main study material (some supporting material is available in Additional reading page).
%   \item The book can be recommended as a reference material in later use.
%   \item The exam questions are based only on the book (3rd edition)
%   \item Lectures support the written material.
%     \begin{itemize}
%     \item The lectures are interactive
%     \item The lectures may contain additional interesting material not
%       included in the book.
%     \end{itemize}
%   \end{itemize}

%   \note{none}

% \end{frame}



\begin{frame}
  \frametitle{Bayesian data analysis}  %
  \framesubtitle{Exercises}
  \begin{itemize}
  \item Weekly exercises introduced on Monday lecture
  \item Related R/Python demos available
  \item TAs available on Thursday 12--16 and Friday 10--12
  \item Exercise deadlines on Sunday
  \item After exercise deadline grading period Monday--Tuesday
  \item Students grade 4 other exercises using peergrade.io
  \end{itemize}

\end{frame}

\begin{frame}
  \frametitle{Exercises}  %
  \framesubtitle{peergrade.io}
  \begin{itemize}
  \item Used in BDA course since 2016
  \item Each student grades 4 exercises (randomly distributed)
  \item Detailed grading instructions
  \item Also text feedback
  \item Possible to flag inappropriate grading
  \item TAs check flagged gradings and strongly coflicting gradings
  \item Possible to give thumb up for great feedback
    \begin{itemize}
    \item those who give good feedback will get bonus points
    \end{itemize}
  \end{itemize}
  
\end{frame}

\begin{frame}
  \frametitle{Exercises}  %
  \framesubtitle{peergrade.io}

  \begin{itemize}
  \item Combined score: 70\% submission performance, 30\% feedback performance
    \pause
  \item Hand-in score:
    \begin{itemize}
    \item averaging the scores from peers
    \item after flagging teacher may overrule the score
    \item different exercises have different weight
    \end{itemize}
    See details at \url{http://help.peergrade.io/interfaces-and-features/grading-and-scores/the-hand-in-score}
    \pause
  \item Feedback score:
    \begin{itemize}
    \item The constructive score
    \item The hand-in evaluation accuracy score
    \item The feedback evaluation accuracy score
    \item The feedback completeness score
    \item The feedback evaluation completeness score
    \end{itemize}
    See details at \url{http://help.peergrade.io/interfaces-and-features/grading-and-scores/the-feedback-score}
  \end{itemize}
  
\end{frame}

\begin{frame}
  \frametitle{Project work}  %
  \framesubtitle{}
  \begin{itemize}
  \item Project work in groups of 2--3
    \begin{itemize}
    \item combines all the pieces in one project work
    \item R or Python notebook report
    \item project report peer graded
    \item oral presentation 
    \end{itemize}
  \end{itemize}
  
\end{frame}

% \begin{frame}
%   \frametitle{Bayesian data analysis}  %
%   \framesubtitle{Exercises}
%   \begin{itemize}
%   \item Exercises related to the current topic are introduced Mondays (and available at MyCourses)
%   \item Assistants help with exercises Thursdays 12-16 at Maari-C. Fridays 10-12 at Y342a/Maari-C
%   \item Deadline: the exercise solutions have to be returned before the next lecture Monday 10am
%   \item For example, first exercises will presented 21th September and
%     need to be returned 28th September (but sometimes there is more than 1 week time)
%   \end{itemize}
% \end{frame}

% \begin{frame}
%   \frametitle{Bayesian data analysis}  %
%   \framesubtitle{Estimated workload}
%   \begin{itemize}
%   \item Total 140 hours, $140/27\approx5$ credit points.
%     \begin{itemize}
%     \item Lectures 10*2=20 hours.
%     \item Exercise sessions 7*2=14 hours
%     \item Reading the book, approx 200 pages. Assuming 5 pages per hour = 40 hours
%     \item Doing the exercises, 7*8 = 56 hours.
%     \item Exam and studying for the exam, 10 hours.
%     \end{itemize}
%   %\item The exam will be planned such that doing the exercises is the best way to prepare oneself for the exam.
%   \end{itemize}
% \end{frame}

% \begin{frame}
%   \frametitle{Bayesian data analysis}  %
%   \framesubtitle{Changes from the previous year}
%   \begin{itemize}
%   \item Based on student feedback, following changes from the previous year will be implemented.
%     \begin{itemize}
%     \item The exercises will be organized into 7 separate exercise sets instead of a single project report that must be returned at the end of the course.
%     \item Basically the same exercises; however, the exact problem formulations will be re-written to some exercises.
%     \item Paper and pencil solutions will be accepted to those exercises that are straightforward calculations.
%     \end{itemize}    
%   \end{itemize}
% \end{frame}

\end{document}

%%% Local Variables:
%%% mode: latex
%%% TeX-master: t
%%% End:
