\documentclass[t]{beamer}
%\documentclass[finnish,english,handout]{beamer}

\usepackage[T1]{fontenc}
\usepackage[utf8]{inputenc}
\usepackage{newtxtext} % times
%\usepackage[scaled=.95]{cabin} % sans serif
\usepackage{amsmath}
\usepackage[varqu,varl]{inconsolata} % typewriter
\usepackage[varg]{newtxmath}
\usefonttheme[onlymath]{serif} % beamer font theme
\usepackage{microtype}
\usepackage{afterpage}
\usepackage{url}
\urlstyle{same}
% \usepackage{amsbsy}
% \usepackage{eucal}
\usepackage{rotating}
\usepackage{listings}
\usepackage{lstbayes}
\usepackage[all,poly,ps,color]{xy}
\usepackage{eurosym}

\usepackage{natbib}
\bibliographystyle{apalike}

% minted
\usepackage{minted}
\setminted{highlightcolor=yellow!25}
\newmintinline{r}{}
% The following is adjusted from
% https://tex.stackexchange.com/questions/548592/changing-all-colors-to-black-white-using-minted-sty
\makeatletter
\newcommand{\minted@style@bw}{%
  \renewcommand\fcolorbox[3][]{##3}%
  \renewcommand\textcolor[3][]{##3}%
  \color{gray}
}
% define new minted option "gray"
\minted@def@opt@switch{gray}
\fvset{formatcom*={%
  \ifthenelse{\equal{\minted@get@opt{gray}{true}}{true}}
  {\minted@style@bw}{}%
}}
\makeatother
% The following is ajusted from
% https://tex.stackexchange.com/questions/74459/remove-space-before-colorbox
\newcommand{\reducedstrut}{\vrule width 0pt height .9\ht\strutbox depth .9\dp\strutbox\relax}
\newcommand{\highlight}[1]{%
  \begingroup
  \setlength{\fboxsep}{0pt}%  
  \colorbox{yellow!30}{\reducedstrut\detokenize{#1}\/}%
  \endgroup
}

\mode<presentation>
{
  \setbeamercovered{invisible}
  \setbeamertemplate{itemize items}[circle]
  \setbeamercolor{frametitle}{bg=white,fg=navyblue}
  \setbeamertemplate{navigation symbols}{}
  \setbeamertemplate{headline}[default]{}
  \setbeamertemplate{footline}[split]
  % \setbeamertemplate{headline}[text line]{\insertsection}
  \setbeamertemplate{footline}[frame number]
}

\pdfinfo{            
  /Title      (BDA, Lecture 10) 
  /Author     (Aki Vehtari) % 
  /Keywords   (Bayesian data analysis)
}

\definecolor{forestgreen}{rgb}{0.1333,0.5451,0.1333}
\definecolor{navyblue}{rgb}{0,0,0.5}
\definecolor{list1}{rgb}{0,0.2549,0.6784}
\renewcommand{\emph}[1]{\textcolor{navyblue}{#1}}
\definecolor{set11}{HTML}{E41A1C}
\definecolor{set12}{HTML}{377EB8}
\definecolor{set13}{HTML}{4DAF4A}

\graphicspath{{./figs/}}


\parindent=0pt
\parskip=8pt
\tolerance=9000
\abovedisplayshortskip=0pt

%\renewcommand{\itemsep}{0pt}
% Lists
\newenvironment{list1}{
   \begin{list}{$\color{list1}\bullet$}{\itemsep=6pt}}{
  \end{list}}
\newenvironment{list1s}{
  \begin{list}{$\includegraphics[width=5pt]{logo.eps}$}{\itemsep=6pt}}{
  \end{list}}
\newenvironment{list2}{
  \begin{list}{-}{\baselineskip=12pt\itemsep=2pt}}{
  \end{list}}
\newenvironment{list3}{
  \begin{list}{$\cdot$}{\baselineskip=15pt}}{
  \end{list}}

\def\o{{\mathbf o}}
\def\t{{\mathbf \theta}}
\def\w{{\mathbf w}}
\def\x{{\mathbf x}}
\def\y{{\mathbf y}}
\def\z{{\mathbf z}}

\def\peff{p_{\mathrm{eff}}}
\def\eff{\mathrm{eff}}

\DeclareMathOperator{\E}{E}
\DeclareMathOperator{\Var}{Var}
\DeclareMathOperator{\var}{var}
\DeclareMathOperator{\Sd}{Sd}
\DeclareMathOperator{\sd}{sd}
\DeclareMathOperator{\Gammad}{Gamma}
\DeclareMathOperator{\Invgamma}{Inv-gamma}
\DeclareMathOperator{\Bin}{Bin}
\DeclareMathOperator{\Negbin}{Neg-bin}
\DeclareMathOperator{\Poisson}{Poisson}
\DeclareMathOperator{\Beta}{Beta}
\DeclareMathOperator{\logit}{logit}
\DeclareMathOperator{\N}{N}
\DeclareMathOperator{\U}{U}
\DeclareMathOperator{\normal}{normal}
\DeclareMathOperator{\R2D2}{R2D2}
\DeclareMathOperator{\BF}{BF}
\DeclareMathOperator{\Invchi2}{Inv-\chi^2}
\DeclareMathOperator{\NInvchi2}{N-Inv-\chi^2}
\DeclareMathOperator{\InvWishart}{Inv-Wishart}
\DeclareMathOperator{\tr}{tr}
% \DeclareMathOperator{\Pr}{Pr}
\def\euro{{\footnotesize \EUR\, }}
\DeclareMathOperator{\rep}{\mathrm{rep}}

% \def\dashxy(#1){%
%   /xydash{[#1] 0 setdash}def}
% \def\grayxy(#1){%
%   /xycolor{#1 setgray}def}
% \newgraphescape{D}[1]{!{\ar @*{[!\dashxy(2 2)]} "#1"}}
% \newgraphescape{P}[1]{!{\ar "#1"}}
% \newgraphescape{F}[1]{!{*+=<2em>[F=]{#1}="#1"}}
% \newgraphescape{O}[1]{!{*+=<2em>[F]{#1}="#1"}}
% \newgraphescape{V}[1]{!{*+=<2em>[o][F]{#1}="#1"}}
% \newgraphescape{B}[3]{!{{ "#1"*+#3\frm{} }.{ "#2"*+#3\frm{} } *+[F:!\grayxy(0.75)]\frm{}}}


\title[]{Bayesian data analysis}
\subtitle{}

\author{Aki Vehtari}

\institute[Aalto]{}
 
\date[]{}

%\beamerdefaultoverlayspecification{<+->}

\begin{document}

\begin{frame}{R2D2 prior}

  \begin{list1}
  \item Useful for models with many predictors
  \end{list1}
\end{frame}

\begin{frame}{$R^2$, proportion of variance explained}

  \vspace{-\baselineskip}
    \begin{equation*}
      \frac{\text{explained variance}}{\text{total variance}}
    \end{equation*}

    \uncover<2->{
  \includegraphics[width=5.3cm]{fake_yhat.pdf}
  \includegraphics[width=5.3cm]{fake_res.pdf}
}
  
  \vspace{-\baselineskip}
  \begin{align*}
    \uncover<3->{R^2 }& \uncover<3->{= \frac{{\Var}_{\mu}}{{\Var}_{\mu}+{\Var}_{\mathrm res}}, \text{ where } \mu = \hat{y}, \text{and res}=y-\hat{y}}  \\
  & \uncover<4->{  \approx \frac{2.4}{2.4+0.85} \approx 0.74}
  \end{align*}
  
  
\end{frame}

\begin{frame}{LOO-$R^2$}

  Replace posterior mean prediction with LOO-CV predictions

    \uncover<2->{
  \includegraphics[width=5.3cm]{fake_yloo.pdf}
  \includegraphics[width=5.3cm]{fake_loores.pdf}
}
  
  \vspace{-\baselineskip}
  \begin{align*}
    \uncover<3->{R^2 }& \uncover<3->{\approx \frac{2.4}{2.4+0.85} \approx 0.74}\\
    \uncover<3->{\mathrm{LOO-}R^2 }& \uncover<3->{\approx \frac{2.4}{2.4+1.0} \approx 0.7}
  \end{align*}

  \hfill
  
  {\color{gray}\small Code and examples \url{avehtari.github.io/bayes_R2/bayes_R2.html}}
\end{frame}

\begin{frame}{Bayesian-$R^2$}

  \vspace{-\baselineskip}
  Model based, with posterior

    \uncover<2->{
      \includegraphics[width=5.3cm]{fake_mus.pdf}
\uncover<3->{
      \includegraphics[width=5.3cm]{fake_bayes_R2.pdf}
  }
    }
    
  
    \uncover<2->{
  \vspace{-3\baselineskip}
  \begin{align*}
 {\Var}_{\mu}^s & = V_{n=1}^N \hat{y}_n^s\\
    {\Var}_{\mathrm res}^s & = \Var(y \mid x, \alpha^s, \beta^s, \sigma^s) = (\sigma^2)^s 
  \end{align*}
}

\vspace{-2\baselineskip}
  \begin{itemize}
  \item<4-> Posterior presents the uncertainty
  \item<5-> Posterior mean of Bayesian-$R^2$ is optimistic compared to LOO-$R^2$
  \item<6-> Can also be used to examine $R^2$ prior distribution
  \end{itemize}

  {\color{gray}\footnotesize \href{https://doi.org/10.1080/00031305.2018.1549100}{Gelman, Goodrich, Gabry, and Vehtari (2019). R-squared for Bayesian regression models. \textit{The American Statistician}, 73(3):307-309.}}

\end{frame}

\begin{frame}{Implied prior on $R^2$}

  \begin{itemize}
  \item Draw from the model parameter priors
  \item Compute prior predictions
  \item Compute Bayesian-$R^2$ using the prior draws
  \end{itemize}
  
\end{frame}

\begin{frame}[fragile]{Implied prior on $R^2$}

  Regression and Other Stories, Section 12.7 Models for regression
  coefficients: 26 covariates normalized to have mean 0 and sd 1
  
  \only<1>{
    Independent normal priors for coefficients
    \begin{align*}
      \beta & \sim \normal(0, 2.5) \\
      \sigma & \sim t_3(0, 3)
    \end{align*}
    
    \vspace{-4\baselineskip}
    \includegraphics[width=10cm]{student_bayes_R2_n1p.pdf}
  }
  \only<2>{
    Scaled normal priors for coefficients
    \begin{align*}
      \beta & \sim \normal(0, 2.5/\sqrt{26}) \\
      \sigma & \sim t_3(0, 3)
    \end{align*}

    \vspace{-4\baselineskip}
    \includegraphics[width=10cm]{student_bayes_R2_n2p.pdf}
  }
  \only<3>{
    R2D2 prior
    \begin{align*}
      \beta & \sim \R2D2(1/3, 3, 1/2, \sigma) \\
      \sigma & \sim t_3(0, 3)
    \end{align*}

    \vspace{-4\baselineskip}
    \includegraphics[width=10cm]{student_bayes_R2_mp.pdf}
  }
  
\end{frame}

\begin{frame}[fragile]{R2D2 prior}

  \begin{itemize}
  \item Define the prior on $R^2$
  \item Joint prior on coefficients which depends also on $\sigma$
  \item D2 refers to Dirichlet decomposition
  \item<2-> With mean 1/3 and precision 3, R2D2 prior states smaller
    $R^2$ are more likely a priori, but locally the slope is not steep
  \item<3-> Often good prior when there are many covariates
  \item<4-> Predictively consistent prior: adding more covariates
    doesn't change the prior on $R^2$
  \end{itemize}

  \hfill

  {\footnotesize\color{gray}\href{https://doi.org/10.1080/01621459.2020.1825449}{Zhang, Naughton, Bondell, and Reich
      (2022). Bayesian regression using a prior on the model fit: The {R2-D2} shrinkage prior. \textit{Journal of the American Statistical
        Association}, 117:862--874.}\\
    \href{https://doi.org/10.1214/23-EJS2136}{Aguilar and Bürkner (2023). Intuitive joint priors for {Bayesian} linear multilevel models: {The} {R2D2M2} prior. \textit{Electronic Journal of Statistics}, 17:1711--1767.}}
  
\end{frame}

\begin{frame}[fragile]{Priors and posteriors}


\only<1>{\includegraphics[width=11cm]{fig-implied-R2-prior-posterior-loo-html-1.pdf}}
\only<2>{\includegraphics[width=11cm]{fig-implied-R2-prior-posterior-loo-html-2.pdf}}
\only<3>{\includegraphics[width=11cm]{fig-implied-R2-prior-posterior-loo-html-3.pdf}}
\only<4>{\includegraphics[width=11cm]{fig-implied-R2-prior-posterior-loo-html-4.pdf}}
  
\end{frame}

\begin{frame}[fragile]{brms}

\begin{minted}{r}
fit <- brm(y ~ .,
           prior = prior(R2D2(mean_R2 = 1/3,
                              prec_R2 = 3,
                              cons_D2 = 1/2),
                         class=b)))
\end{minted}  

  \vspace{2\baselineskip}

  \uncover<2->{
  See the assignment notebook for additional $R^2$ related code.
}


\end{frame}

\end{document}

%%% Local Variables: 
%%% TeX-PDF-mode: t
%%% TeX-master: t
%%% TeX-command-extra-options: "-shell-escape"
%%% End: 
